% Copyright 2004 by Till Tantau <tantau@users.sourceforge.net>.
%
% In principle, this file can be redistributed and/or modified under
% the terms of the GNU Public License, version 2.
%
% However, this file is supposed to be a template to be modified
% for your own needs. For this reason, if you use this file as a
% template and not specifically distribute it as part of a another
% package/program, I grant the extra permission to freely copy and
% modify this file as you see fit and even to delete this copyright
% notice. 

\documentclass[10pt, mathserif, profesionalfont]{beamer}

\usepackage[utf8]{inputenc}
\usepackage[spanish]{babel}



\usepackage{amsmath,amsfonts,amssymb,amsthm}
\usepackage{booktabs}
\usepackage{hyperref}
\usepackage{authblk}

\usepackage[acronym]{glossaries}

\renewcommand\Affilfont{\itshape\small}
\renewcommand\Authand{ y }

\newtheorem{thm}{Teorema}

\usepackage{array}
\newcolumntype{L}[1]{>{\raggedright\let\newline\\\arraybackslash\hspace{0pt}}p{#1}}
\newcolumntype{C}[1]{>{\centering\let\newline\\\arraybackslash\hspace{0pt}}p{#1}}
\newcolumntype{R}[1]{>{\raggedleft\let\newline\\\arraybackslash\hspace{0pt}}p{#1}}

\newcolumntype{X}[1]{>{\raggedright\let\newline\\\arraybackslash\hspace{0pt}}m{#1}}
\newcolumntype{Y}[1]{>{\centering\let\newline\\\arraybackslash\hspace{0pt}}m{#1}}
\newcolumntype{Z}[1]{>{\raggedleft\let\newline\\\arraybackslash\hspace{0pt}}m{#1}}

\usepackage{color}
\makeglossaries

\newacronym{ndtm}{NDTM}{Máquina de Turing No Determinista}

\newacronym{sat}{SAT}{SATISFACTIBILIDAD}

\newacronym{3sat}{3SAT}{3-SATISFACTIBILIDAD}

\newacronym{vc}{VC}{VERTEX COVER}

% There are many different themes available for Beamer. A comprehensive
% list with examples is given here:
% http://deic.uab.es/~iblanes/beamer_gallery/index_by_theme.html
% You can uncomment the themes below if you would like to use a different
% one:
%\usetheme{AnnArbor}
%\usetheme{Antibes}
%\usetheme{Bergen}
%\usetheme{Berkeley}
%\usetheme{Berlin}
%\usetheme{Boadilla}
%\usetheme{boxes}
%\usetheme{CambridgeUS}
%\usetheme{Copenhagen}
%\usetheme{Darmstadt}
%\usetheme{default}
%\usetheme{Frankfurt}
%\usetheme{Goettingen}
%\usetheme{Hannover}
%\usetheme{Ilmenau}
%\usetheme{JuanLesPins}
%\usetheme{Luebeck}
\usetheme{Madrid}
%\usetheme{Malmoe}
%\usetheme{Marburg}
%\usetheme{Montpellier}
%\usetheme{PaloAlto}
%\usetheme{Pittsburgh}
%\usetheme{Rochester}
%\usetheme{Singapore}
%\usetheme{Szeged}
%\usetheme{Warsaw}

\title{VERTEX COVER}

% A subtitle is optional and this may be deleted
\subtitle{Seis problemas básicos $\mathcal{NP}$-completos}

\author{Ángel David Martín Rodríguez, Víctor David Mayorca Luis \and Jesús Eduardo Plasencia Pimentel}
% - Give the names in the same order as the appear in the paper.
% - Use the \inst{?} command only if the authors have different
%   affiliation.

\institute[ULL]{Universidad de La Laguna} % (optional, but mostly needed)

% - Use the \inst command only if there are several affiliations.
% - Keep it simple, no one is interested in your street address.

\date{Curso 2016/2017}
% - Either use conference name or its abbreviation.
% - Not really informative to the audience, more for people (including
%   yourself) who are reading the slides online

\subject{Theoretical Computer Science}
% This is only inserted into the PDF information catalog. Can be left
% out. 

% If you have a file called "university-logo-filename.xxx", where xxx
% is a graphic format that can be processed by latex or pdflatex,
% resp., then you can add a logo as follows:

% \pgfdeclareimage[height=0.5cm]{university-logo}{university-logo-filename}
% \logo{\pgfuseimage{university-logo}}

% Delete this, if you do not want the table of contents to pop up at
% the beginning of each subsection:
\AtBeginSubsection[]
{
  \begin{frame}<beamer>{Outline}
    \tableofcontents[currentsection,currentsubsection]
  \end{frame}
}

% Let's get started
\begin{document}

\begin{frame}
  \titlepage
\end{frame}



\section{Introduction}


\begin{frame}{Problemas involucrados}

\begin{block}{\gls{3sat}}
{\small
\noindent ENTRADA: Un conjunto de cláusulas $C=\left \{c_1, c_2, \dots, c_m \right \}$ sobre un conjunto finito $U$ de variables tal que $|c_i|=3$, para $1\le i \le m$.

\noindent PREGUNTA: ¿Existe una asignación booleana para $U$, tal que satisfaga todas las cláusulas de $C$? 
}
\end{block}

\begin{block}{\gls{vc}}
{\small 
\noindent ENTRADA: Un grafo $G = (V,E)$ y un entero positivo $K < |V|$.

\noindent PREGUNTA: ¿Existe un recubrimiento de vértices de tamaño $K$ o menos para $G$, esto es, un subcojunto $V' \subseteq V$ tal que $|V'|\le K$ y, por cada eje $\{u,v\} \in E$, al menos uno de $u$ y $v$ pertenezcan a $V$?
}
\end{block}

\end{frame}

\section{Demostración de NP-completitud}

\begin{frame}{\gls{vc} es NP-completo I}
    
\begin{block}{\gls{vc} $\in \mathcal{NP}$}    
Es fácil comprobar que \gls{vc}	$\in \mathcal{NP}$, ya que se puede encontrar una algoritmo para una \gls{ndtm} que reconozca el lenguaje $L(\mbox{VC},e)$, para un esquema de codificación $e$, en un número de pasos acotado por una función polinomial.

\end{block}

\end{frame}

\begin{frame}{VC es NP-completo II}
    
\begin{block}{3SAT $\preceq$ VC}
Sea $U=\{u_1,u_2,\ldots,u_n\}$ y $C=\left \{c_1, c_2, \dots, c_m \right \}$ una instancia de 3SAT. Debemos construir un grafo $G=(V,E)$ y un entero positivo $K \le |V|$ tal que $G$ posee un recubrimiento de vertices de tamaño $K$ o menos si y solo si C es satisfactible.
\end{block}

\end{frame}


\begin{frame}{VC es NP-completo III}
    
\begin{block}{3SAT $\preceq$ VC}    
Para cada variable $u_i \in U$ hay un componente asignador $T_i=(V_i,E_i)$ con $V_i=\{u_i, \bar{u_i}\}$ y $E_i=\{\{u_i, \bar{u_i}\}\}$, esto es, dos vertices unidos por un solo eje. Se debe notar que cualquier recubrimiento de vértices debe de contener por lo menos uno de $u_i$ o $\bar{u_i}$ a manera de recubrir el único eje en $E_i$.
\end{block}

\end{frame}


\begin{frame}{VC es NP-completo IV}

\begin{block}{3SAT $\preceq$ VC}    
Para cada cláusula $c_j \in C$ hay un componente de prueba de satisfacción $S_j=(V'_J,E'_j)$, que consiste de tres nodos y tres aristas que unen los nodos a manera de triángulo.
\[V'_j=\{a_1[j],a_2[j],a_3[j]\}\]
\[E'_j=\{\{a_1[j],a_2[j]\},\{a_1[j],a_3[j]\},\{a_2[j],a_3[j]\}\}\]
Se debe notar que cualquier recubrimiento de vértices debe contener como mínimo dos v'ertices de $V'_j$ de manera de cubrir todos las aristas en $E'_j$.
\end{block}

\end{frame}


\begin{frame}{VC es NP-completo V}

\begin{block}{3SAT $\preceq$ VC}    
La única parte de la construcción que depende de qué literales aparecen en que cláusulas en las colección de aristas de comunicación. Para cada cláusula $c_j\in C$ sean los tres literales en $c_j$ denotados como $x_j$, $y_j$ y $z_j$. Entonces los ejes de comunicación que parten de $S_j$ están dados por:
\[E''_j=\{ \{a_1[j],x_j\}, \{a_2[j],y_j\}, \{a_3[j],z_j\} \}\]
\end{block}

\end{frame}


\begin{frame}{VC es NP-completo VI}

\begin{block}{3SAT $\preceq$ VC}    
La construcción de nuestra instancia de VC se completa fijando $K=n+2m$ y $G=(V,E)$ donde
\[V=(\bigcup^n_{i=1}V_i)\cup (\bigcup^m_{j=1}V'_j)\]
y
\[E=(\bigcup^n_{i=1}E_i)\cup (\bigcup^m_{j=1}E'_j)\cup (\bigcup^m_{j=1}E''_j)\]
\end{block}

\end{frame}


\begin{frame}{VC es NP-completo VII}

\begin{block}{3SAT $\preceq$ VC}    
Es trivial ver como la construcción es posible de ser realizada en tiempo polinomial. Solo resta demostrar que $C$ es satisfactible si y solo si $G$ posee un recubrimiento de vértices de como mucho tamaño $K$.
\end{block}

\end{frame}


\begin{frame}{VC es NP-completo VIII}

\begin{block}{3SAT $\preceq$ VC}    
Primero, supongamos que $V' \subseteq V$ es un recubrimiento de vértices para $G$ con $|V'| \le K$. Basado en lo dicho anteriormente $V'$ debe contener al menos un vértice de cada $T_i$ y al menos dos vértices de cada $S_j$. Podemos entonces utilizar la manera en la que V' intersacta cada componente de asignación para obtener una asignación boolena $t: U \rightarrow \{T,F\}$. Simplemente fijamos $t(u_i)=T$ si $u_i \in V'$ y $t(u_i) = F$ si $\bar{u_i}\in V'$
\end{block}

\end{frame}


\begin{frame}{VC es NP-completo IX}

\begin{block}{3SAT $\preceq$ VC}    
Para ver que esta asignación booleana satisface cada una de las cláusulas $c_j \in C$, considera los tres ejes en $E''_j$. Solo dos de estos ejes pueden ser cubiertos con vértices de $V'_j \cap V'$, así que al menos uno de ellos debe estar cubierto por un vértices de algún $V_i$ que pertenece a $V'$. Pero esto implica que el literal correspondiente, ya sea $u_i$ o $\bar{u_i}$, de la cláusula $c_j$ es verdadero bajo la asignación boolena $t$ y por lo tanto la cláusula $c_j$ es satisfactible por $t$. Debido a que esto es cierto por cada $c_j \in C$, sigue que $t$ es una asignación satisfactoria para C.
\end{block}

\end{frame}

\begin{frame}{VC es NP-completo X}

\begin{block}{3SAT $\preceq$ VC}    
supongamos que $t: U \rightarrow {T,F}$ es una asignación satisfactoria para $C$. El recubrimiento de vértices correspondiente $V'$ incluye un vértice de cada $T_i$ y dos vértices de cada $S_j$. El vértice de $T_i$ en $V'$ es $u_i$ si $t(u_i)=T$ y es $\bar{u_i}$ si $t(u_i)=F$. Esto asegura que al menos uno de los tres ejes de cada conjunto $E''_j$ esté cubierto, puesto que $t$ satisface cada cláusula $c_j$. Por lo tanto necesitamos solo incluir en $V'$ los puntos finales de $S_j$ de los otros dos vértics en $E''_j$ (el cual puede o no estar cubierto por otros vértices de los componentes de asignación) y esto nos da el recubrimiento de vértices deseado.
\end{block}

\end{frame}
\end{document}